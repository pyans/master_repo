\documentclass[a4j,12pt,twoside]{jreport}

\usepackage{subfiles}
%\usepackage{graphicx}
\usepackage[dvipdfmx]{graphicx}
\usepackage{kit-is-m}

% textlint-disable sentence-length

%% 下のコメントで囲まれた範囲はtextlintによるチェックを無視する
%% ルール名を省略すると全て無視
%
%  % textlint-disable ルール名
%  文章
%  % textlint-enable ルール名

% textlint-enable sentence-length

%% Uncomment the following line if you prefer to make the English version cover page.
%\EngCover{1}
\author{工繊 太郎}
%% \eauthor または \eauthori のどちらかを使うこと.
%% 日本人学生: 名-姓の順で定義する.
%% International students: Use \eauthori macro ONLY(Comment \eauthor macro out).
%%                         Write first and last names exactly as shown on 
%%                         your residence card.
\eauthor{Taro}{Kousen} % 日本人学生(名-姓の順であることに注意!)
%\eauthori{John Smith} % for International students
\idnumber{17622999}
\deadline{2020}{2}{10} % 提出日{西暦年}{月}{日}
%% 長いタイトルで途中改行する場合,改行したい位置で
%% \coverbreak, \abstbreak, \\ のいずれかを使う.
%%   \coverbreak :表紙でのみこの位置で改行(line break here in coverpage only)
%%    \abstbreak :概要でのみこの位置で改行(line break here in abstract only)
%%            \\ :表紙・概要ともにこの位置で改行
%%【例】
%% \title{長い長いタイトル\coverbreak の改行位置\abstbreak に関する研究}
%% \etitle{Very long title\coverbreak long long\abstbreak long title}
\title{修士論文・卒業研究報告書の書き方}
\etitle{How to write the Master's Thesis}
%% 以下,指導教員名の前の数字は職名を表す
%% 1:教授,2:准教授,3:講師,4:助教,5:助手
%% The first parameter describes a job classification.
%% 1:Prof., 2:Assoc.Prof., 3:Lecturer, 4:Assist.Prof., 5: Assist.
\advisor{1}{工繊 一郎} % 主任指導教員
\secondadvisor{2}{情報 次郎} % (必要ならば)指導教員(if needed)
%\thirdadvisor{4}{情報 三郎} % (必要ならば)指導教員(if needed)
\begin{document}
\maketitle

%%
%% 和文概要(400字程度)
%%
\subfile{sections/abstract-ja}

% textlint-disable sentence-length

%% 英文概要(150〜200 words)
%% English Abstract (150-200 words)
%%
%% 【注意】
%% このスタイルファイルが生成する英文概要は,学務課が要求する
%% 書式と厳密には一致しない.
%% 学務課へは別途 Word ファイル(Doc形式)の提出が必要なので,
%% Word 上で最終印刷することを勧める.
%% Please print out the final edition of the abstract by using MS Word 
%% for submitting to the Educational Affairs. You also have to submit 
%% the Word file online.

% textlint-enable sentence-length
\subfile{sections/abstract-en}

% 目次
% Index
\begin{contents}
\tableofcontents % 目次の作成
%\nchapter{記号説明} % 必要ならば
%% ここに記号説明を書く
\end{contents}

% 緒言
% Introduction
\subfile{sections/introduction}

% 論文の体裁
% Format of thesis
\subfile{sections/format}

% 論文の構成
% Chapters
\subfile{sections/chapters}

% 結言
% Conclusions
\subfile{sections/conclusions}

% 謝辞
% Acknowledgement
\subfile{sections/acknowledgement}

% 参考文献
% References
\bibliographystyle{kit-is} % 文献スタイルファイル
\bibliography{main} % 文献データベース

% 付録
% Appendix
\subfile{sections/appendix}

\end{document}

