%! TEX root = ../main.tex
\documentclass[main]{subfiles}

\begin{document}

\chapter{論文の体裁}
\section{基本方針}
論文の体裁は,原則として電子情報通信学会(信学会)和文論文誌Dの投稿規定に
準拠する.
情報処理学会の投稿規定に準拠してもよいが,信学会規定との混在は認めない.
修士論文や卒業研究報告書に固有の事項は本報告書を基準とする.

\section{冊子体裁}
冊子体裁の詳細を表~\ref{tab1}に示す.
表紙・裏表紙と製本テープは研究室を通じて配布されるものを使う.
卒業研究報告書は,簡易製本冊子のみ提出し,CD-R等は添付しない(別途研究室
で保管する).
修士論文は,論文pdfファイルや関連データを収録したCD-Rと,簡易製本冊子の
両方を提出する.

\begin{table}[p]
\begin{center}
\caption{修士論文・卒業研究報告書の冊子体裁一覧}
\label{tab1}
\begin{tabular}{p{5.5zw}|p{29zw}} \Hline
項目 & 内容 \\ \hline
判型と形態 & A4縦長,横書き \\ \hline
記述の分量 & 原則として,図表ページを除く本文20ページ以上とする. \\ \hline
ページ設定 & 全角37文字×30行/ページ,1段組,天地左右に各25mmの余白を
  とる.\\ \hline
フォント & 本文12pt,章・節の見出しや表紙は本報告書に準じて適宜大きく
  取る.\\ \hline
ノンブル(ページ番号) & 
紙面の下中央に通し番号のみをつける.括弧やハイフンはつけない.
表紙,概要にはつけない.目次(記号説明含む)はローマ数字(小文字)を使い,
本文+図表の最初から付録の末尾までは算用数字を使う. \\\hline
用紙 & 表紙,裏表紙は色上質紙(卒業研究報告書は淡青色,修士論文は淡緑色),
それ以外は普通紙に印刷する. \\\hline
印刷 & 原則モノクロ印刷.表紙と概要は片面印刷.それ以外は両面印刷. 
目次が奇数枚の場合は,目次最終ページの裏面をノンブルのみの空白ページとし,
本文は用紙を改めて印刷する.章を奇数ページから始める必要はないので,本文
には空白ページを挿入しないこと.\\ \hline
製本 & 左側3カ所をホチキス留めし,それを覆うように小口から製本テープで
留めて簡易製本する.2穴パンチ位置(中央 $\pm$ 40mm)へのホチキスは避ける.
パンチ穴は開けないこと.\\ \Hline
\end{tabular}
\end{center}
\end{table}



\section{用字と用語}
本節では用字と用語\footnote{電子情報通信学会和文論文誌投稿のしおり(情報・
	システムソサイエティ) 2.4用字と用語についてを引用.}について説明
	する.

% textlint-disable no-doubled-conjunction

\begin{itemize}
\item[(a)] 用字は原則として「常用漢字」を用い,仮名は「新仮名づかい」と
	   する(付録F)\footnote{http://www.ieice.org/jpn/shiori/pdf/furoku\_f.pdf}.
\item[(b)] 用語は原則として,
(1)「文部省学術用語集,電気工学編」及び信学会編
(2)「電子情報通信用語辞典」,
(3)「電子情報通信ハンドブック」によるものとする.
\item[(c)] 量記号,単位記号の略号(SI)及びシンボルは,原則として信学会編「電
      子情報通信ハンドブック」によるものとする.
\item[(d)] 句読点は,句点「.」と読点「,」をそれぞれ全角で用いる.
\end{itemize}

% textlint-enable no-doubled-conjunction

\section{数式}
数式は\LaTeX のディスプレイ数式モード, Office2003のMicrosoft数式,
Word2007の数式エディタ(組み立て型式)などを用いて組版する.
数式は文章中には極力埋め込まず独立した行に中央揃えもしくは等号揃えで
書く.
分数を1/2と書いたり,和記号の上下限を $\Sigma$ の右に書くなど,いわゆる
インライン型式は使わない.
式(\ref{eqn1})に示す通り,数式には(章番号.通し番号)という書式の式番号を
右端に付し,本文で引用する.

\begin{equation}
y(n)=\sum_{k=-\infty}^{\infty}x(k)h(n-k)\label{eqn1}
\end{equation}

\section{図,写真,表}
図,写真(図に準拠),表は原則として著者が独自に作成したもののみを使用する.
文献,書籍,WWW等からの転載(スキャンして貼り付ける行為)は厳禁する.
著作物を引用する場合は出所を明らかにし,著作権を侵害しないよう注意する.

図のうち,線画はPowerPoint, Illustrator, PiC\TeX, PSTricks等のドロー系
ソフト等を用いて丁寧に描く.
グラフはExcelやGnuplot等を用いて作成する.
図には図[章番号].[章内の図番号] [図の見出し] という書式の題名
(キャプション)を図の下側に付す.
図~\ref{fig1} のように図中に複数の要素が含まれる場合,各要素に
(a), (b), (c), … で始まる副見出しをつける.

表は列見出しを必須とし,必要に応じて行見出しをつける.
表組みの罫線はなるべく少なくする.
左右端の縦罫線は引かない.
天地の横罫線は太罫線もしくは二重罫線を引く.
表には表[章番号].[章内の表番号] [表の見出し] という書式の題名を
表の上側に付す.

信学会和文論文誌Dの投稿規定では,図表には英文表題を併記するとあるが,
修士論文・卒業研究報告書には,英文表題は必要ない.

カラー画像やディスプレイ画面など,カラーでなければ表現できない場合を除き,
図表はモノクロで作図する.
例えば,複数の折れ線グラフを描く場合,各折れ線をカラーで区別するのでは
なく,線種で区別する.

\section{脚注}
本文中で説明すると文脈が乱れるときは,説明を脚注に書く.
長くなる場合は付録とする.
脚注はページ最下段に1本横線を引き,その下に
 (注[通し番号(算用数字)]):○○○○ と書く.
本文中の参照箇所にも(注[通し番号])を上つきで付す.

Wordで組版する場合など,(注[通し番号])の自動付番が困難であれば,
本報告書のように\footnote{これは\LaTeX で組版しているので,正しい脚注に
	なっている.},
左端および本文注の参照箇所に,通し番号のみを
上付き算用数字で付す書式を用いてもよい.

\end{document}
