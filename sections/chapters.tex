%! TEX root = ../main.tex
\documentclass[main]{subfiles}

\begin{document}

\chapter{プログラム構成}

\section{環境、使用言語}
本研究はvisual studio code(以下vscode)の拡張機能として実装する.
% この理由は(ここに理由を入力)だからである.
vscodeの拡張機能はtypescriptを用いて実装する.
環境としてJavaScript実行環境のNode.js,
及びjavascriptのAPIであるvscode APIを用いる.ディレクトリ構成を以下に示す.

\section{アルゴリズム}
本拡張機能の動作としては主に3つの段階に分けて考えることができる.
すなわち,現在のソースコードを抽象構文木化し,jsonファイルに変換して内容を取り込む入力部分,
抽象構文木化したソースコード群を比較評価し,現在のカーソル部分と類似した部分木を見つけ出す比較部分,
見つけた部分木から対応するヒントコードを表示する出力部分である.

\begin{description}
\item[1.入力部分 ]  \\*
入力として学習者の現在編集しているコードファイル(以下元ファイルと称す),及び元ファイル上でのカーソル現在位置,
すなわちコマンド呼び出し時にカーソルのあった行番号を取得する.
学習者は通常,現在編集しているコードの続きについてヒントを求めていると考えられるので
カーソル部分及びその前部分のコードをプログラム上で注目する.

コードの抽象構文木化にはC言語のコンパイラであるclangの機能を用いる.
clangにはコンパイルするソースファイルの抽象構文木を出力する機能があり,さらに結果をJSON形式で出力することもできる.
これを用い,コマンド呼び出し時に保存した学習者のソースファイルをclangを用いて抽象構文木化し
このJSONファイルを読み込んでプログラム上で扱う.
また,比較に用いる過去の回答コードもclangを用いて抽象構文木化しておく.こちらは今回手動で行った.
\item[2.比較部分 ]  \\*
入力部分の項でも述べたように,まずは抽象構文木化した学習者のコードファイルを調査し
コマンド呼び出し時のカーソル部分のノードを検索する.
本研究ではコードのパターンマッチにおいて「注目箇所及び前後部分の兄弟ノードが一致する部分木を検索する」
ことによって類似コードを見つけ出すという手法をとる.そしてパターンがマッチした場合
そのマッチの具合に応じて点数をつけ,点数が一番高いものをヒントコードとして推薦する.
\item[3.出力部分 ]  \\*
マッチしたコードの前後を取り出し,出力する.
\end{description}

ああああ
\end{document}