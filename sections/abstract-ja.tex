%! TEX root = ../main.tex
\documentclass[main]{subfiles}

\begin{document}

\begin{abstract}
修士論文や卒業研究報告書の執筆は,修士および学部における学修の総仕上げで
あり,そのねらいは,専門知識の俯瞰と体系化,論理的な文章の記述力養成,
時間制約の下で計画的に物事を進める能力の養成,にある.
本報告書は,修士論文・卒業研究報告書の体裁や構成を説明し,書き方の見本を
示すことを目的とする.
修士論文や卒業研究報告書は,いわゆる学術論文に準ずるので,「全ての主張や
評価が根拠に基づいている」ことが重要である.
そのような文章には,読み手が把握しやすい適切な書き方があるが,それを各自
勝手に工夫する,いわゆる車輪の再発明の無駄を避ける点で,体裁や構成を
定める意義がある.
冊子体裁,用字と用語,数式,図・写真・表,脚注など内容に依らない「体裁」
については,原則として電子情報通信学会和文論文誌Dの投稿規定に準拠する.
情報処理学会の投稿規定に準拠してもよいが,信学会規定との混在は認めない.
\end{abstract}

\end{document}
