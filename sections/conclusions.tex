%! TEX root = ../main.tex
\documentclass[main]{subfiles}

\begin{document}

\chapter{結言}
修士論文・卒業研究報告書の書き方を,体裁と構成の観点から詳しく説明した.
学生諸君は,必ず全文通読した上で執筆にあたって頂きたい. 
最後に,論文や報告書の執筆にあたって推奨される行動を紹介し,結言とする.

\begin{enumerate}
\item 執筆の時間とエネルギーを十分確保する.
\item 体調管理に留意する.睡眠時間と3度の食事を削ってはならない.
\item \TeX, Word, Gnuplot, Excel等のソフトウェアの使い方に習熟する.
式番号,図表番号の付け方,相互参照,目次の自動生成,線画の作成,グラフの
見出し,軸線,目盛,プロット記号,フォントの変更方法など,論文や報告書
作成に特有の手法を予め習得する.
\item 電子情報通信学会のWWWページにアクセスし,「和文論文誌投稿のしおり
(情報・システムソサイエティ)」および「論文の書き方と査読の方法」を熟読する.
\item 書き上がってから指導教員に見せる前に,最低3回は推敲(読み返して
文章を改善すること)を繰り返す.
ディスプレイで見ずに印刷して赤ペンを持ち声に出して読んでみるとよい.
\item 原稿や図表は日付とバージョン番号をつけて管理し,古いバージョンの
ファイルも残す.
\end{enumerate}

\end{document}

