%! TEX root = ../main.tex
\documentclass[main]{subfiles}

\begin{document}

\chapter{緒言}
近年プログラミングの学習においては,iSnapなどに代表される自動フィードバック機能,
すなわち学習者の回答に応じて自動生成されたヒントを提供する機能を持つシステムの
研究が進められている.
適切に用いられるフィードバックは学習者の学習効果,モチベーションを向上させることや
教師の負担を減らすことが期待される.

% textlint-disable sentence-length

このようなフィードバックシステムにおいては,ヒントコードを生成するために
ネットワークを用いた経路探索などの複雑なアルゴリズムを用いるものもある.
本論文では複雑なアルゴリズムを用いず,C言語を対象とした簡易な自動フィードバックのアルゴリズムを制作し,
どの程度フィードバックが正確になされたかについて調査する.

方法としては学習者のソースコードを抽象構文木に変換,過去の学生の回答データから
学習者の編集部分にマッチする部分木を検索,一番望ましいと思われる部分木をコードに再変換して表示する.
このツールを用いて幾らかのテストケースを検証,提案されたヒントがどの程度役に立つものか調査する.

本報告書の構成を紹介する.まず,本研究の背景や目的などについてより詳細に記述する.
次に,本研究において開発した簡易的なフィードバックツールについて,
言語等の環境やアルゴリズムについて詳細に説明する.
その次に,上記フィードバックツールにおいて行った調査の詳細について記述する.
最後に,行った調査の結果,及びその考察について解説する.

\end{document}
