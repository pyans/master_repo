\begin{table}[p]
\begin{center}
\caption{修士論文・卒業研究報告書の冊子体裁一覧}
\label{tab1}
\begin{tabular}{p{5.5zw}|p{29zw}} \Hline
項目 & 内容 \\ \hline
判型と形態 & A4縦長,横書き \\ \hline
記述の分量 & 原則として,図表ページを除く本文20ページ以上とする. \\ \hline
ページ設定 & 全角37文字×30行/ページ,1段組,天地左右に各25mmの余白を
  とる.\\ \hline
フォント & 本文12pt,章・節の見出しや表紙は本報告書に準じて適宜大きく
  取る.\\ \hline
ノンブル(ページ番号) & 
紙面の下中央に通し番号のみをつける.括弧やハイフンはつけない.
表紙,概要にはつけない.目次(記号説明含む)はローマ数字(小文字)を使い,
本文+図表の最初から付録の末尾までは算用数字を使う. \\\hline
用紙 & 表紙,裏表紙は色上質紙(卒業研究報告書は淡青色,修士論文は淡緑色),
それ以外は普通紙に印刷する. \\\hline
印刷 & 原則モノクロ印刷.表紙と概要は片面印刷.それ以外は両面印刷. 
目次が奇数枚の場合は,目次最終ページの裏面をノンブルのみの空白ページとし,
本文は用紙を改めて印刷する.章を奇数ページから始める必要はないので,本文
には空白ページを挿入しないこと.\\ \hline
製本 & 左側3カ所をホチキス留めし,それを覆うように小口から製本テープで
留めて簡易製本する.2穴パンチ位置(中央 $\pm$ 40mm)へのホチキスは避ける.
パンチ穴は開けないこと.\\ \Hline
\end{tabular}
\end{center}
\end{table}

